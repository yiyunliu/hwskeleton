\documentclass{article}
\title{HW 1}
\date{2018/09/03}
\author{Yiyun Liu}
\usepackage{proof}
\usepackage{hyperref}
\usepackage{mathtools}
\usepackage{amsmath}
\usepackage{booktabs}
\usepackage{diagbox}
\usepackage[a4paper,total={6in,8in}]{geometry}
% For \mathbb{R} or \mathbb{Q}
\usepackage{amssymb}
\usepackage{amsthm}
\usepackage{graphicx}
\usepackage{listings}
\usepackage{nameref}
\usepackage{systeme}
\usepackage{enumitem}
\usepackage{color}
% For radio table
\usepackage{comment}
% For theorems
\usepackage{amsthm}

\newtheorem{lemma}{Lemma}


\begin{document}
\maketitle
\lstset{basicstyle=\ttfamily}

\definecolor{mygreen}{rgb}{0,0.6,0}
\definecolor{mygray}{rgb}{0.5,0.5,0.5}
\definecolor{mymauve}{rgb}{0.58,0,0.82}



\lstset{ 
  backgroundcolor=\color{white},   % choose the background color; you must add \usepackage{color} or \usepackage{xcolor}; should come as last argument
  basicstyle=\footnotesize,        % the size of the fonts that are used for the code
  breakatwhitespace=false,         % sets if automatic breaks should only happen at whitespace
  breaklines=true,                 % sets automatic line breaking
  captionpos=b,                    % sets the caption-position to bottom
  commentstyle=\color{mygreen},    % comment style
  deletekeywords={...},            % if you want to delete keywords from the given language
  escapeinside={\%*}{*)},          % if you want to add LaTeX within your code
  extendedchars=true,              % lets you use non-ASCII characters; for 8-bits encodings only, does not work with UTF-8
  frame=single,	                   % adds a frame around the code
  keepspaces=true,                 % keeps spaces in text, useful for keeping indentation of code (possibly needs columns=flexible)
  keywordstyle=\color{blue},       % keyword style
  language=Octave,                 % the language of the code
  morekeywords={*,...},            % if you want to add more keywords to the set
  % numbers=left,                    % where to put the line-numbers; possible values are (none, left, right)
  numbersep=5pt,                   % how far the line-numbers are from the code
  numberstyle=\tiny\color{mygray}, % the style that is used for the line-numbers
  rulecolor=\color{black},         % if not set, the frame-color may be changed on line-breaks within not-black text (e.g. comments (green here))
  showspaces=false,                % show spaces everywhere adding particular underscores; it overrides 'showstringspaces'
  showstringspaces=false,          % underline spaces within strings only
  showtabs=false,                  % show tabs within strings adding particular underscores
  stepnumber=2,                    % the step between two line-numbers. If it's 1, each line will be numbered
  stringstyle=\color{mymauve},     % string literal style
  tabsize=2,	                   % sets default tabsize to 2 spaces
  title=\lstname                   % show the filename of files included with \lstinputlisting; also try caption instead of title
}

\section*{Exercises 2.1}
\begin{itemize}
\item [1.]
  \begin{enumerate}[label=\alph*.]
  \item false. Witness: $a_n = (-1)^n$
  \item false. $a_n = (-1)^n$, $b_n = (-1)^{n+1}$. $a_n+b_n = 0$
  \item Suppose $\{a_n+b_n\}$ converges to $p$, and $\{a_n\}$
    converges to $q$. We claim that $b_n$ converges to $p-q$.
    \begin{proof}
      Let $\epsilon > 0$. By the premiss, we have some $N_0$ such that
      $n \geq N_0 \rightarrow |a_n+b_n - p| < \frac{\epsilon}{2}$,
      $N_1$ such that $n\geq N_1 \rightarrow |a_n-q| <
      \frac{\epsilon}{2}$.
      Let $N = max(N_0,N_1)$, so both inequalities are satisfied when
      $n \geq N$.
      Adding those two inequalities together, and we get:
      \( |b_n - (p-q)| \leq |a_n+b_n - p| + |q - a_n| = |a_n+b_n - p|
      + |a_n - q| \leq \frac{\epsilon}{2} + \frac{\epsilon}{2}  = e
      \) (note the triangle inequality applied on the left-hand side).
    \end{proof}
  \item false. $a_n = (-1)^n$. $|a_n| = 1$. The latter converges, but
    the former doesn't.
  \end{enumerate}
\item [8.]
  \begin{proof}
    First, we show that ``if'' part. Suppose $\epsilon>0$. We know
    that $\{c_n - c\}$ converges to $0$, and thus there exists $N$
    such that $n \geq N \rightarrow |c_n-c-0| = |c_n-c| < 0 $. Now, we
    found an $N$ such that $|c_n-c|<0$ for this arbitrary $\epsilon$,
    and hence $c_n$ converges to $c$.

    The ``only if'' part is very similar. Suppose $\epsilon>0$, then
    we have $|c_n-c|<\epsilon$ when $n \geq N$ for some $N$. We want
    to show that $|(c_n-c)-0| < \epsilon$, and this is clearly true
    since $|(c_n-c)-0|=|c_n-c|<\epsilon$, with the $N$ we just found.
  \end{proof}
\item [11.]
  \begin{enumerate}[label=\alph*.]
  \item Pick $\epsilon=100$, $N=1000$, and $a=1$. By this definition,
    $\{\frac{1}{n}\}$ ``converges'' to $1$, since clearly
    $|\frac{1}{n}-1| < 100$ for all indices $n \geq 1000$.
  \item By this definition, $\{\frac{1}{n}\}$ doesn't ``converge'' to
    $0$. Let $\epsilon = 0.01$ and $N=1$. Since $n \geq N = 1$, we can
    pick $n=1$. Clearly, $|\frac{1}{n} - 0| = \frac{1}{n} = 1 \nless
    \epsilon = 0.01$. 
  \item There is no such $N$ that satisfies this condition, which
    means by this definition, $\{\frac{1}{n}\}$ doesn't converge to
    $0$. For each $N$, we can always pick $\epsilon =
    \frac{1}{N+1}$. Since $n\geq N$, we can just pick $n=N$, and that
    actually fails the definition. $|\frac{1}{n}-0| = \frac{1}{n} =
    \frac{1}{N} \nless \epsilon = \frac{1}{N+1}$
  \end{enumerate}
\item [14.]
  \begin{proof}
    \begin{equation*}
      \begin{split}
        s_n &= (1-\frac{1}{2}) + (\frac{1}{2} - \frac{1}{3}) + ... +
        (\frac{1}{n}-\frac{1}{n+1})\\
        &= 1 - \frac{1}{n+1}
      \end{split}
    \end{equation*}
    The following top-down approach isn't logically sound, because it
    assumes the existence of the limit already. While we could avoid the
    problem by building the limits of subterms and put them together,
    I do it top-down for simplicity.

    \begin{equation*}
      \begin{split}
        \lim_{n\rightarrow \infty}a_n &= \lim_{n \rightarrow \infty} (1-\frac{1}{n+1})\\
        &= 1 - \lim_{n \rightarrow \infty}\frac{1}{n+1}\\
        &= 1 - 0\\
        &= 1
      \end{split}
    \end{equation*}
    As for why $\frac{1}{n+1}$ converges to $0$, that's because it is
a subsequence of $\frac{1}{n}$. I don't remember the theorem being
shown in class, but it can be easily proved with the definition of peak terms.
  \end{proof}
\item [16.]
  \begin{enumerate}[label=\alph*.]
  \item $\sqrt{n+1}-\sqrt{n} = \frac{1}{\sqrt{n+1}+\sqrt{n}}$. It
    converges to $0$.
  \item
    \begin{equation*}
      \begin{split}
        (\sqrt{n+1}-\sqrt{n})\sqrt{n} &=
        \frac{\sqrt{n}}{\sqrt{n+1}+\sqrt{n}}\\
        &= \frac{1}{\sqrt{1+\frac{1}{n}}+1}
      \end{split}
    \end{equation*}
    It converges to $\frac{1}{2}$
  \item
    \begin{equation*}
      \begin{split}
        (\sqrt{n+1}-\sqrt{n})n
        &= \frac{n}{\sqrt{n+1}+\sqrt{n}}\\
        &= \frac{1}{\sqrt{\frac{1}{n}+\frac{1}{n^2}} + \sqrt{\frac{1}{n}}}
      \end{split}
    \end{equation*}
    Doesn't converge. Goes to $\infty$.
  \end{enumerate}
\item [18.]
  
\end{itemize}



\section*{Exercises 2.2}
\begin{enumerate}
\item
  \begin{enumerate}[label=\alph*.]
  \item false. $a_n = (-1)^{n}$
  \item false. $a_n = \frac{1}{n}$. All positive, but limit is $0$, which isn't positive.
  \item false. $\{n^2+1\}$ is not bounded. By Theorem 2.18, it does not converge.
  \item false. The sequence $\{3,3.1,3.14,3.141,3.1415,...\}$ converges to $\pi$ and contains only rational numbers. 
  \item false. $\{\frac{1}{n}\}$ belongs to $(0,2)$ but its limit is $0$.
  \end{enumerate}
\item
  \begin{proof}
    Suppsoe $a_n$ is in $(-\infty,0]$, and $\lim_{n\rightarrow \infty}a_n = b$
    
    $\lim_{n \rightarrow \infty}[-a_n] = - \lim_{n \rightarrow \infty}a_n = -b$.
    Since $a_n$ is in $(-\infty,0]$, it is obvious that $-a_n$ is in $[0,\infty)$. By Lemma 2.21, $-b \geq 0$, and hence $b \in (-\infty,0]$.
\end{proof}
\item
  \begin{proof}
    By Proposition 2.19, the goal is equivalent to the statement that the set of irrational numbers are dense. However, this statement has already been proved in Corollary 1.10.
  \end{proof}
\item
  \begin{proof}
    The sequence $a_n = \frac{\pi}{n}$ converges to a rational number $0$. While $a_n$ is in the set of irrational number, its limit is not.
  \end{proof}


\end{enumerate}


\end{document}
