\documentclass{article}
\title{HW 1}
\date{2018/09/03}
\author{Yiyun Liu}
\usepackage{proof}
\usepackage{hyperref}
\usepackage{mathtools}
\usepackage{amsmath}
\usepackage{booktabs}
\usepackage{diagbox}
\usepackage[a4paper,total={6in,8in}]{geometry}
\usepackage{amsthm}
\usepackage{graphicx}
\usepackage{listings}
\usepackage{nameref}
\usepackage{systeme}
\usepackage{enumitem}
% For \mathbb{R}, real number symbol
\usepackage{amssymb}
\begin{document}
\maketitle
\lstset{basicstyle=\ttfamily}

\section*{Appendix A}
\begin{itemize}
\item [4.] Rewrite in standard notation, and our goal is to prove
  \[-(ab^{-1})=(-a)b^{-1}=a(-b)^{-1}\]
  \begin{proof}
    \begin{equation*}
      \begin{split}
        ab^{-1}+(-a)b^{-1}
        &= b^{-1}a + b^{-1}(-a) \textbf{ (Commutativity)} \\
        &= b^{-1}(a+(-a)) \textbf{ (Distributivity)}\\
        &= b^{-1}\cdot 0 \textbf{ (Definition of inverse)}\\
        &= 0 \textbf{ (Proposition A.1)}
      \end{split}
    \end{equation*}
    Therefore $-(ab^{-1})=(-a)b^{-1}$, by the fact that additive
    inverse is unique.\\
    \begin{equation*}
      \begin{split}
        ab^{-1}+a(-b)^{-1}
        &= a(b^{-1}+(-b)^{-1}) \textbf{ (Distributivity)}\\
        &= a(b^{-1} -(b^{-1})) \textbf{ (Proposition A.4 ii)}\\
        &= a\cdot 0\\
        &= 0 \textbf{ (Proposition A.1)}
      \end{split}
    \end{equation*}
  \end{proof}
  Therefore $a(-b)^{-1}=-(ab^{-1})$.\\
  By transitivity of $=$, \(-(ab^{-1})=(-a)b^{-1}=a(-b)^{-1}\)
\end{itemize}

\section*{Chapter 1}
\begin{itemize}
\item [1.]
  \begin{enumerate}[label=\alph*.]
  \item false. $1 \notin \mathbb{R}-\mathbb{Q}$
  \item false. Since the set is inductive, $\mathbb{N}$ has to be a
    subset of it by definition. $2\in \mathbb{N}$ but there is no $x
    \in \mathbb{Q}$ such that $x^2=2$. Contradiction.
  \item false. $\sqrt{2},-\sqrt{2} \in \mathbb{R}-\mathbb{Q}$ but
    $\sqrt{2}-\sqrt{2}=0\in\mathbb{Q}$
  \item $\sqrt{2}\in \mathbb{R}-\mathbb{Q}$ but
    $\sqrt{2}\cdot\sqrt{2}=2\in \mathbb{Q}$
  \item true.
    \begin{proof}
      Assume the negation holds.
      There exists $n$ such that $n^2$ is odd but $n$ is even.
      $n^2 = 2k$ for some $k \in \mathbb{Z}$,
      $n^2=4k^2=2\cdot(2k^2)$. $n$ is even. Contradiction\\
    \end{proof}
  \end{enumerate}
\item [2.]
  \begin{enumerate}[label=\alph*]
  \item false. $sup$ not necessarily belongs to the set, e.g. $(0,1)$.
  \item true. $0$ is a lower bound of $S$. By definition $0 \leq
    \inf{S}$
  \item true. Clearly, $\sup{S}$ is an upper bound of $B$, then by
    definition $\sup{B} \leq \sup{S}$.
  \end{enumerate}

\item [5.]
  Our goal is to prove $\forall n
  \sum_{i=1}^{n}i^3=(\sum_{i=1}^ni)^2$.
  \begin{proof}
    \begin{description}
    \item[Base case]
      When $n=1$
      \[LHS=\sum_{i=1}^1i^3=1\]
      \[RHS=(\sum_{i=1}^1i)^2=1^2=1\]
      $LHS=RHS$, the proposition holds.
    \item[Inductive case]
      Suppose the proposition holds for $P(n)$ where $n\geq 1$. 
      \begin{equation*}
        \begin{split}
          \sum_{i=1}^{n+1}i^3
          &= (\sum_{i=1}^{n}i^3) + (n+1)^3\\
          &= (\sum_{i=1}^{n}i)^2 + (n+1)^3 \textbf{ (Inductive
            hypothesis)}\\ 
          &= (\frac{n(n+1)}{2})^2 + (n+1)^3 \textbf{ (Example 1.1)}\\
          &= \frac{n^2(n+1)^2}{4} + (n+1)^3\\
          &= \frac{(n+1)^2(n^2+4(n+1))}{4}\\
          &= \frac{(n+1)^2(n+2)^2}{4}\\
          &= (\frac{(n+1)((n+1)+1)}{2})^2\\
          &= (\sum_{i=1}^{n+1}i)^2
        \end{split}
      \end{equation*}
    \end{description}
  \end{proof}
\item [7.]
  \begin{enumerate}[label=\alph*.]
  \item
    \begin{proof}
      Suppose the negation holds. That is, there exists such $r \in
      \mathbb{Q}$ and $c \in \mathbb{R}-\mathbb{Q}$. Then we have $p
      \in \mathbb{Z}$, $q \in \mathbb{Z}-\{0\}$ such that $r =
      \frac{p}{q}$ (it should be obvious what $p$ and $q$ belongs to
      and that they're closed under certain operations. I'll omit
      those later to be concise).
      Since $r+c$ is rational, we have $r+c =
      \frac{p}{q}+c=\frac{p'}{q'}$ for some $p'$, $q'$.\\
      Subtracting $r$ to the other side and we get $c =
      \frac{qp'-q'p}{qq'}$. Clearly the numerator and denominator are
      both integers (the denominator being non-zero).\\
      $c$ is rational. Contradiction.
    \end{proof}

  \end{enumerate}
\item Suppose the negation holds. Then there exists rational $r=\frac{p}{q}$
  and irrational $c$ such that $rc = \frac{p'}{q'}$ for some $p'$,
  $q'$.
  $c=\frac{qp'}{pq'}$ where $qp'$ is integer and $pq'$ is a non-zero
  integer. $c$ is rational. Contradiction.
\end{itemize}
\end{document}
