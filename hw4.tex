\documentclass{article}
\title{HW 4}
\date{2018/10/03}
\author{Yiyun Liu}
\usepackage{proof}
\usepackage{hyperref}
\usepackage{mathtools}
\usepackage{amsmath}
\usepackage{booktabs}
\usepackage{diagbox}
\usepackage[a4paper,total={6in,8in}]{geometry}
% For \mathbb{R} or \mathbb{Q}
\usepackage{amssymb}
\usepackage{amsthm}
\usepackage{graphicx}
\usepackage{listings}
\usepackage{nameref}
\usepackage{systeme}
\usepackage{enumitem}
\usepackage{color}
% For radio table
\usepackage{comment}
% For theorems
\usepackage{amsthm}

\newtheorem{lemma}{Lemma}

\begin{document}
\maketitle
\lstset{basicstyle=\ttfamily}

\definecolor{mygreen}{rgb}{0,0.6,0}
\definecolor{mygray}{rgb}{0.5,0.5,0.5}
\definecolor{mymauve}{rgb}{0.58,0,0.82}



\lstset{ 
  backgroundcolor=\color{white},   % choose the background color; you must add \usepackage{color} or \usepackage{xcolor}; should come as last argument
  basicstyle=\footnotesize,        % the size of the fonts that are used for the code
  breakatwhitespace=false,         % sets if automatic breaks should only happen at whitespace
  breaklines=true,                 % sets automatic line breaking
  captionpos=b,                    % sets the caption-position to bottom
  commentstyle=\color{mygreen},    % comment style
  deletekeywords={...},            % if you want to delete keywords from the given language
  escapeinside={\%*}{*)},          % if you want to add LaTeX within your code
  extendedchars=true,              % lets you use non-ASCII characters; for 8-bits encodings only, does not work with UTF-8
  frame=single,	                   % adds a frame around the code
  keepspaces=true,                 % keeps spaces in text, useful for keeping indentation of code (possibly needs columns=flexible)
  keywordstyle=\color{blue},       % keyword style
  language=Octave,                 % the language of the code
  morekeywords={*,...},            % if you want to add more keywords to the set
  % numbers=left,                    % where to put the line-numbers; possible values are (none, left, right)
  numbersep=5pt,                   % how far the line-numbers are from the code
  numberstyle=\tiny\color{mygray}, % the style that is used for the line-numbers
  rulecolor=\color{black},         % if not set, the frame-color may be changed on line-breaks within not-black text (e.g. comments (green here))
  showspaces=false,                % show spaces everywhere adding particular underscores; it overrides 'showstringspaces'
  showstringspaces=false,          % underline spaces within strings only
  showtabs=false,                  % show tabs within strings adding particular underscores
  stepnumber=2,                    % the step between two line-numbers. If it's 1, each line will be numbered
  stringstyle=\color{mymauve},     % string literal style
  tabsize=2,	                   % sets default tabsize to 2 spaces
  title=\lstname                   % show the filename of files included with \lstinputlisting; also try caption instead of title
}


\section*{Ex 8}
\begin{itemize}
\item [24.] This is isomorphic to the Klein four-group: $\mathbb{Z}_2 \times \mathbb{Z}_2$. We've shown in the first midterm that it has four elements, and it is clearly not cyclic. Since there are only two groups of order $2$, we know it has to be the non-cyclic one. An isomorphism contains the following mappings:
  id $\rightarrow (0,0)$, rotate 180 degress $\rightarrow (0,1)$, vertical flip $\rightarrow (1,1)$, horizontal flip $\rightarrow (1,0)$. 
\item [46.]
  Because $n \geq 3$, $
  \begin{pmatrix}
    1 & 2
  \end{pmatrix}
$ and $
\begin{pmatrix}
  1 & 2 & 3
\end{pmatrix}
$ are elements of $S_n$. However,
$
\begin{pmatrix}
  1 & 2
\end{pmatrix}
\begin{pmatrix}
  1 & 2 & 3
\end{pmatrix} =
\begin{pmatrix}
  2 & 3
\end{pmatrix}
\neq
\begin{pmatrix}
  1 & 3
\end{pmatrix}
=
\begin{pmatrix}
  1 & 2 & 3
\end{pmatrix}
\begin{pmatrix}
  1 & 2
\end{pmatrix}
$. Therefore, $S_n$ is not abelian when $n \geq 3$.
\end{itemize}

\section*{Ex 9}
\begin{itemize}
\item [11.]
  \begin{equation*}
    \begin{pmatrix}
      1 & 3 &  4
    \end{pmatrix}
    \begin{pmatrix}
      2 & 6
    \end{pmatrix}
    \begin{pmatrix}
      5 & 8 & 7
    \end{pmatrix}
  \end{equation*}
  \begin{equation*}
    \begin{pmatrix}
      1 & 4
    \end{pmatrix}
    \begin{pmatrix}
      1 & 3
    \end{pmatrix}
    \begin{pmatrix}
      2 & 6
    \end{pmatrix}
    \begin{pmatrix}
      5 & 7
    \end{pmatrix}
    \begin{pmatrix}
      5 & 8
    \end{pmatrix}
  \end{equation*}
\item [13.]
  \begin{enumerate}[label=\alph*.]
  \item 4
    
  \item A cycle of size $n$ has order $n$. I define size as the number of
    entries within the parentheses. 
  \item $6$,$4$. Computed by taking the least common multiple of the
    size of each disjoint cycle.
  \item
    \begin{itemize}
    \item [10.]
      \begin{equation*}
        \begin{pmatrix}
          1 & 8
        \end{pmatrix}
        \begin{pmatrix}
          3 & 6 & 4
        \end{pmatrix}
        \begin{pmatrix}
          5 & 7
        \end{pmatrix}
      \end{equation*}
      Least common multiple: 6.
    \item [11.]
      \begin{equation*}
        \begin{pmatrix}
          1 & 3 & 4
        \end{pmatrix}
        \begin{pmatrix}
          2 & 6
        \end{pmatrix}
        \begin{pmatrix}
          5 & 8 & 7
        \end{pmatrix}
      \end{equation*}
      Least common multiple: 6.
    \item [12.]
      \begin{equation*}
        \begin{pmatrix}
          1 & 3 & 4 & 7 & 8 & 6 & 5 & 2
        \end{pmatrix}
      \end{equation*}
      This is kind of degenerate. The least common multiple of a
      single number is just itself, in this case: 8. Or we can just
      use part b.
    \end{itemize}

  \item Given a finite permutation written in the form of multiplication of
    disjoint cycles, the order of the permutation is equal to the
    least common multiple of the sizes of cycles. 
  \end{enumerate}
\item [23.]
  \begin{itemize}
  \item [a.] false. $
    \begin{pmatrix}
      1 & 2
    \end{pmatrix}
    \begin{pmatrix}
      3 & 4
    \end{pmatrix}
    $ in $S_4$ can't be written as a cycle.
  \item [i.] true. If we fix 1 element in $S_8$, we are effectively
    permuting only $7$ elements.
  \end{itemize}
\item [27.]
  \begin{enumerate}[label=\alph*.]
  \item
    \begin{proof}
      Let's start with cycles first.  We know from the textbook that a
      cycle $
      \begin{pmatrix}
        a_1 & a_2 &\ldots &a_k
      \end{pmatrix}
      $ can be written as
      $ \underbrace{ \begin{pmatrix} a_1 & a_k \end{pmatrix}
        \ldots \begin{pmatrix} a_1 & a_2 \end{pmatrix}}_{k-1}$.  We
      also know that every permutation in $S_n$ can be written as
      multiplications of disjoint cycles. Since the cycles are
      disjoint, the total size of the cycles must be less than or
      equal to $n$.

      Let $\tau$ be an arbitrary permutation in $S_n$. We write $\tau$
      in the form of $k$ disjoint cycles, and let
      $m=n_1 + n_2 + \ldots + n_k$ be the total size of the cycles,
      where $n_i$ denotes the size of the $i$th cycle. $m \leq n$ as
      discussed above. The $i$th cycle can be further written as
      multiplication $n_i - 1$ transpositions. After rewriting, we get
      $(n_1 -1) + \ldots + (n_k -1) = (n_1 + \ldots n_k) - k = m - k$
      transpositions. When $k=0$, the permutation is identity, so we
      can say it's a product of $0$ transpositions. When $k > 0$,
      $m - k < m \leq n$. $m - k < n \rightarrow m - k \leq n -1$.

      We've written $\tau$ as multiplication of no more than $n-1$
      transpositions. 
    \end{proof}
  \item
    \begin{proof}
      Continue the proof above. When the permutation is not cyclic, we
      have $k >1$. $m - k < m - 1 \leq n -1$. $m - k < n -1
      \rightarrow m - k \leq n -2$. Remember $m-k$ is the number of
      transpositions after rewriting the permutation into
      multiplication of transpositions. 
    \end{proof}
\end{enumerate}

\end{itemize}

\section*{Ex 10}
\begin{itemize}
\item [4.]
  $\{0,4,8\}$,$\{1,5,9\}$,$\{2,6,10\}$,$\{3,7,11\}$
\item [6.]
  Basically copying down the two columns and removing the duplicates:
  $\{\rho_0,\mu_2\}$, $\{\rho_1,\delta_2\}$,$\{\rho_2,\mu_1\}$,$\{\rho_3,\delta_1\}$
\end{itemize}



\end{document}
