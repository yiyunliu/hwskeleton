\documentclass{article}
\title{HW 4}
\date{2018/09/21}
\author{Yiyun Liu}
\usepackage{proof}
\usepackage{hyperref}
\usepackage{mathtools}
\usepackage{cancel}
\usepackage{amsmath}
\usepackage{booktabs}
\usepackage{diagbox}
\usepackage[a4paper,total={6in,8in}]{geometry}
% For \mathbb{R} or \mathbb{Q}
\usepackage{amssymb}
\usepackage{amsthm}
\usepackage{graphicx}
\usepackage{listings}
\usepackage{nameref}
\usepackage{systeme}
\usepackage{enumitem}
\usepackage{color}
% For radio table
\usepackage{comment}
% For theorems
\usepackage{amsthm}

\newtheorem{lemma}{Lemma}

\begin{document}
\maketitle
\lstset{basicstyle=\ttfamily}

\definecolor{mygreen}{rgb}{0,0.6,0}
\definecolor{mygray}{rgb}{0.5,0.5,0.5}
\definecolor{mymauve}{rgb}{0.58,0,0.82}



\lstset{ 
  backgroundcolor=\color{white},   % choose the background color; you must add \usepackage{color} or \usepackage{xcolor}; should come as last argument
  basicstyle=\footnotesize,        % the size of the fonts that are used for the code
  breakatwhitespace=false,         % sets if automatic breaks should only happen at whitespace
  breaklines=true,                 % sets automatic line breaking
  captionpos=b,                    % sets the caption-position to bottom
  commentstyle=\color{mygreen},    % comment style
  deletekeywords={...},            % if you want to delete keywords from the given language
  escapeinside={\%*}{*)},          % if you want to add LaTeX within your code
  extendedchars=true,              % lets you use non-ASCII characters; for 8-bits encodings only, does not work with UTF-8
  frame=single,	                   % adds a frame around the code
  keepspaces=true,                 % keeps spaces in text, useful for keeping indentation of code (possibly needs columns=flexible)
  keywordstyle=\color{blue},       % keyword style
  language=Octave,                 % the language of the code
  morekeywords={*,...},            % if you want to add more keywords to the set
  % numbers=left,                    % where to put the line-numbers; possible values are (none, left, right)
  numbersep=5pt,                   % how far the line-numbers are from the code
  numberstyle=\tiny\color{mygray}, % the style that is used for the line-numbers
  rulecolor=\color{black},         % if not set, the frame-color may be changed on line-breaks within not-black text (e.g. comments (green here))
  showspaces=false,                % show spaces everywhere adding particular underscores; it overrides 'showstringspaces'
  showstringspaces=false,          % underline spaces within strings only
  showtabs=false,                  % show tabs within strings adding particular underscores
  stepnumber=2,                    % the step between two line-numbers. If it's 1, each line will be numbered
  stringstyle=\color{mymauve},     % string literal style
  tabsize=2,	                   % sets default tabsize to 2 spaces
  title=\lstname                   % show the filename of files included with \lstinputlisting; also try caption instead of title
}


% \section*{Section 2.3}

\begin{enumerate}
\item
  \begin{enumerate}[label=\alph*.]
  \item true.
    \begin{proof}
      A sequence is just a function with type:
      $\mathbb{N} \rightarrow \mathbb{R}$.  Suppose we have
      \[a_n: \mathbb{N} \rightarrow \mathbb{R}\] and
      \[n_k: \mathbb{N} \rightarrow \mathbb{N}\] A subsequence
      $a_{n_{k}}=a_n \circ n_k$. $ran(a_{n_k}) \subseteq ran(a_n)
      \subseteq [-M,M]$ for some $M$.
    \end{proof}
  \item true.
    \begin{proof}
      Continue the convention used in previous problem. $a_n$ is
      monotone, and $n_k$ has to be monotone by
      definition. Composition of monotone function gives us another
      monotone function.
    \end{proof}
  \item true.
    \begin{proof}
      Suppose $a_n$'s limit is $a$.
      For every $\epsilon$, there's an $N$ such that $n \geq N$
      implies $|a_n - a| < \epsilon$. Use the same $N$ for $a_{n_k}$,
      and clearly when $k \geq N$, $a_{n_k}$ the inequality $|a_{n_k}-a| <
      \epsilon$ holds.
    \end{proof}
  \item false. $\{(-1)^n\}$ as counterexample. The even-labeled
    subsequence converges, but the sequence itself doesn't converge.
  \end{enumerate}
\item
  \begin{enumerate}[label=\alph*.]
  \item true.
    \begin{proof}
      Given a sequence in $(0,1)$, we know it has to be bounded. By
      Theorem 2.32 the sequence has a monotone subsequence. This
      subsequence is bounded and monotone, so it converges, by
      monotone convergence theorem.
    \end{proof}
      
    \item false. $\{\frac{1}{n}\}$ converges to $0$.
    \item false. $\{n\}$ is a sequence of rational numbers but it none
      of its subsequences are bounded, and hence do not converge.
      
    \item true.
      \begin{proof}
        Suppose the statement is false. There exists a sequence $a_n$
        in $[0,\infty)$
        such that $\lim_{n\rightarrow \infty}a_n = c < 0$. Then for
        the positive number $-\frac{c}{2}$, there is an $N$ such that
        $n \geq N \rightarrow |a_n - c| < -\frac{c}{2}$. This
        inequality can be rewritten as $-\frac{3}{2}c < a_n <
        \frac{c}{2} < 0$, but we know $a_n \geq 0$ for all $n$. Contradiction.
      \end{proof}
    \item false.
      $\{n\}$ as a counterexample.
  \end{enumerate}
\item $\frac{1}{4},\frac{1}{7},\frac{1}{10},\frac{1}{13},\frac{1}{16}$
\item
  \begin{enumerate}[label=\alph*.]
  \item $1$. The sequence is strictly decreasing, so the first is the
    greatest, and it has to be a peak term.
  \item $2$. We can enumerate the range of the sequence:
    $\{-1,1\}$. $1$ is the greatest value it can take, and the first
    element that's equal to $1$ is the second element of the
    sequence. We also know the first negative element can't be peak
    term because it's smaller than the second one.
  \item Does not exist. Given any $a_n$, $a_n < a_{2n}$. 
  \item $2$. First, we only consider even-labeled terms, because the
    odd-numbered ones are always negative so they are always smaller
    than some positive even-labeled terms. The absolute value of the
    sequence is decreasing, so the largest we can pick is the second.
  \end{enumerate}
\item
  \begin{proof}
    Suppose it does. $\exists n \forall k, k\geq n \rightarrow a_n
    \geq a_k$.

    Apply the fact with $k=n+1$: $a_n \geq a_{n+1}$. $a_n < a_{n+1}$
    as given. Contradiction.
  \end{proof}
\item
  Suppose $\{a_n\}$ is monotonically decreasing. 
  Our goal is to show: $\forall n \in \mathbb{N}, \forall k \in \mathbb{N}, k \geq n
  \rightarrow a_k \leq a_n$.

  Suppose this statement does not hold. We have an $n$ such that
  $\exists k \in \mathbb{N} k \geq n \land a_k > a_n$. It is clear
  that $k > n$ since $k = n$ is impossible with the $a_k > a_n$
  (otherwise $a_k > a_n$ and $k = n$ would lead to contradiction). We
  know that $\{a_n\}$ is monotonically decreasing. Given $k > n$, we
  have $a_k \leq a_n$. However, we just proved that $a_k > a_n$. Contradiction.
  

  
\item [12.]
  \begin{proof}
    I divide this into two cases so we'll get a stronger
    hypothesis for induction. The proof does not reflect my thought
    process. I focused more on formality than clarity.

    When $x_1 \geq  \frac{1+\sqrt{1+4c}}{2}$, we claim that the
    sequence is monotone and bounded below by
    $\frac{1+\sqrt{1+4c}}{2}$.

    First, we show that the sequence is bounded below by
    $\frac{1+\sqrt{1+4c}}{2}$, using induction.

    To prove the base case, we need to show that $x_1 < x_2 = \sqrt{c
      + x_1}$. The premiss tells us that $x_1 \geq
    \frac{1+\sqrt{1+4c}}{2}$. We do the following transformations.
    \begin{equation} \label{eq0}
      \begin{split}
        2x_1 &\geq 1+ \sqrt{1+4c}\\
        2x_1 - 1 &\geq \sqrt{1+4c}\\
        4x_1^2 - 4x_1  + 1 &\geq 1+4c\\
        4x_1^2 - 4x_1 - 4c & \geq 0\\
        x_1^2 - x_1 -c &\geq 0\\
        x_1^2 &\geq x_1 + c\\
        x_1 &\geq \sqrt{x_1+c} = x_2
      \end{split}
    \end{equation}
    The last step, which takes the square root of both sides, is valid
    because the premiss guarantees both sides are positive and the function
    $\lambda x. \sqrt{x}$ preserves order.

    For the inductive case, we assume that the $x_n$ is bounded below
    by $\frac{1+\sqrt{1+4c}}{2}$ for some $n \geq 1$. $x_{n+1} =
    \sqrt{c+x_n} \geq \sqrt{c+\frac{1+\sqrt{1+4c}}{2}} = \sqrt{\frac{1+\sqrt{1+4c}+2c}{2}}$, by inductive
    hypothesis. To prove that $x_{n+1}$ is bounded below by that number,
    we only need to show that $\sqrt{\frac{1+\sqrt{1+4c}+2c}{2}} \geq
    \frac{1+\sqrt{1+4c}}{2}$. Since both sides are positive, it
    suffices to show the equality is preserved when both sides are squared. This can be done by
    subtracting them and show the result is $0$.

    \begin{equation} \label{eq1}
      \begin{split}
        (\sqrt{\frac{1+\sqrt{1+4c}+2c}{2}})^2 -
        (\frac{1+\sqrt{1+4c}}{2})^2 &= \frac{1 + \sqrt{1+4c}+2c}{2} -
        \frac{1+1+4c+2\sqrt{1+4c}}{4}
        \\
        &= \frac{1 + \sqrt{1+4c}+2c}{2} - \frac{1 + \sqrt{1+4c}+2c}{2}\\
        &= 0
      \end{split}
    \end{equation}

    and hence $x_{n+1} =
    \sqrt{c+x_n} \geq \sqrt{\frac{1+\sqrt{1+4c}+2c}{2}} =
    \frac{1+\sqrt{1+4c}}{2}$

    Next, we show that $\{x_n\}$ is monotone decreasing.
    Let $n \geq 1$. The following statements are implicitly connected
    by $\rightarrow$, and the first statement is directly derived the fact that the sequence is bounded below, which
    we just proved.
    \begin{equation} \label{eq2}
      \begin{split}
        x_{n} &\geq \frac{1+\sqrt{1+4c}}{2}\\
        (2x_{n}-1)^2 &\geq 1 + 4c\\
        4x_{n}^2 - 4x_{n} + 1 &\geq 1 + 4c\\
        x_{n}^2-x_{n} -c &\geq 0\\
        x_{n}^2 &\geq x_{n} + c\\
        x_{n} &\geq \sqrt{x_{n} + c} = x_{n+1}
      \end{split}
    \end{equation}

    We have proved that the sequence is monotone and bounded (we have
    only proved the sequence bounded below, but with monotone decreasing
    we can conclude the sequence is bounded above as well). Using the monotone convergence
    theorem, we conclude the sequence converges.

    To find which number it converges to, we use the equality:
    $x_{n+1} = \sqrt{c + x_n}$

    If we think of $\{x_{n+1}\}$ as a subsequence of $\{x_n\}$, they
    clearly converges to the same number. This fact is actually proved
    in one of the exercises.
    \begin{equation*}\label{eq3}
      \begin{split}
        \lim_{n \rightarrow \infty} x_{n} &= \lim_{n \rightarrow \infty}
        x_{n+1}\\
        &= \lim_{n \rightarrow \infty} \sqrt{c +x_n}\\
        &= \sqrt{c + \lim_{n \rightarrow \infty}x_n}
      \end{split}
    \end{equation*}
    Use the shorthand $a=\lim_{n \rightarrow \infty} x_{n}$. We just
    obtained the equation: $a = \sqrt{c + a}$. Solving for $a$, we get
    $a = \frac{1+\sqrt{1+4c}}{2}$. The negative solution can't be the
    limit, because the set $[\frac{1+\sqrt{1+4c}}{2},\infty)$ in which
    our sequence lives is closed.

    
    

    The case where $x_1 \leq \frac{1+\sqrt{1+4c}}{2}$ is almost the dual of
    the case of $\geq$. The proof can be done similarly to the case
    discussed above, by flipping the directions of order relation. To save
    space, I'll go over this case briefly.

    Assuming $x_1 \leq \frac{1+\sqrt{1+4c}}{2}$, we want to prove,
    instead, that $\{x_n\}$ is bounded above by
    $\frac{1+\sqrt{1+4c}}{2}$ and it's monotone increasing.

    The base case is handled as we did in \ref{eq0}, but have $\geq$
    flipped to $\leq$.

    We do the same for the inductive case, flipping the direction for
    the inequality immediately after \ref{eq1}. This works because the
    subtraction gives $0$.

    Flipping the direction of \ref{eq2}, we prove that the sequence is
    monotone increasing.

    The part proving the limit is the solution of the equation is the
    same for both cases.
    
  \end{proof}
\end{enumerate}


\end{document}
