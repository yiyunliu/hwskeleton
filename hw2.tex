\documentclass{article}
\title{HW 2}
\date{2018/09/06}
\author{Yiyun Liu}
\usepackage{proof}
\usepackage{hyperref}
\usepackage{mathtools}
\usepackage{amsmath}
\usepackage{booktabs}
\usepackage{diagbox}
\usepackage[a4paper,total={6in,8in}]{geometry}
% For \mathbb{R} or \mathbb{Q}
\usepackage{amssymb}
\usepackage{amsthm}
\usepackage{graphicx}
\usepackage{listings}
\usepackage{nameref}
\usepackage{systeme}
\usepackage{enumitem}
\begin{document}
\maketitle
\lstset{basicstyle=\ttfamily}

\section*{Exercise 1.2}
\begin{itemize}
\item [1.]
  \begin{enumerate}[label=\alph*.]
  \item false. By proposition 1.6, $\neg \exists n \in \mathbb{Z} \ni n \in (0,1)$
  \item false. $\neg \exists r \in \mathbb{R}^+ \ni r \in (-1,0)$
  \item true. Let $a$ and $b$ be two real numbers such that $a<b$. By the fact that rational number is dense, there is $q \in (a,b)$ where $q$ is rational. If $q$ is not integer, we are done. If it is, let $z = \max(q-1,a)$. $(z,q)$ is a subset of both $(a,b)$ and $(q-1,q)$. By proposition 1.6 and the fact that rational is dense in $\mathbb{R}$, we know that there is some $q' \in (q-1,q) \subseteq (a,b)$ where $q'$ can't be an integer. Done.
  \end{enumerate}
\item [4.]
  \begin{enumerate}[label=\alph*.]
  \item
    \begin{description}
    \item [maximum] 1. $(\frac{1}{n})_{n \in \mathbb{N}}$ is a decreasing sequence. The first element is therefore greatest.
    \item[minimum] Does not exist, by the archimedean property.
    \item[infimum] 0. $0$ is a lower bound, and by archimedean property positive real numbers can't be a lower bound. Thus $0$ is the glb.
    \item[supremum] 1. By definition same as maximum if it exists.
    \end{description}
  \item
    \begin{description}
    \item [maximum] Does not exist. $sup$ is not in the set, as argued below.
    \item [minimum] Does not exist. Similar to maximum.
    \item[infimum] $-\sqrt{2}$ Similar to supremum.
    \item[supremum] $\sqrt{2}$. Suppose it's not. Since real number is dense, we always find a number in between the candidate upper bound and $\sqrt{2}$, but that number is in this set as well, which contradicts the premiss.
    \end{description}
  \end{enumerate}
\item [6.]
  \begin{proof}
    Since $S$ is bounded above by $a$ according to its definition, we know by completeness axiom that $\sup S$ does exist. Now, we will show that $a$ not being the $\sup S$ will lead to a contradiction.

    Suppose $a$ is not $\sup S$. Then there exists $b < a$ such that $b$ is an upper bound of $S$. However, since $\mathbb{Q}$ is dense in $\mathbb{R}$, we know there exists $q \in \mathbb{Q}$ such that $q \in (b,a)$. Clearly, $q \in S$ and $q > b$, which contradicts the assertion that $b$ is an upper bound.
  \end{proof}
\end{itemize}

\section*{Exercise 1.3}
\begin{itemize}
\item [1.]
  \[a^4-b^4 = (a-b)\sum_{k=0}^{3}a^{3-k}b^k\]
  \[a^5-b^5 = (a-b)\sum_{k=0}^{4}a^{4-k}b^k\]
\item [2.]
  \[(a+b)^2=\sum_{k=0}^{2} {2 \choose k} a^{2-k}b^k\]
  \[(a+b)^3=\sum_{k=0}^{3} {3 \choose k} a^{3-k}b^k\]
  \[(a+b)^4=\sum_{k=0}^{4} {4 \choose k} a^{4-k}b^k\]
\item [7.]
  \begin{proof}
  $|a| = |(a+b)+(-b)| \leq  |a+b|+|-b| = |a+b|+|b|$. Thus $|a|-|b| \leq |a+b|$. Apply the fact with $a=b$, $b=a$, we get $|b|-|a| \leq |b+a| \leq |a+b|$, and hence $|a|-|b| \geq -|a+b|$. Apply proposition 1.12 on the two inequalities we obtained, $||a|-|b|| \leq |a+b|$.

  We have shown that $\forall a,b (||a|-|b|| \leq |a+b|)$. Apply the proved statement with $a$ and $-b$, we get $||a|-|-b|| \leq |a-b|$, which is equivalent to $||a|-|b|| \leq |a-b|$
\end{proof}

\end{itemize}


\end{document}
