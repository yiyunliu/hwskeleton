\documentclass{article}
\title{HW 2}
\date{2018/09/06}
\author{Yiyun Liu}
\usepackage{proof}
\usepackage{hyperref}
\usepackage{mathtools}
\usepackage{amsmath}
\usepackage{booktabs}
\usepackage{diagbox}
\usepackage[a4paper,total={6in,8in}]{geometry}
% For \mathbb{R} or \mathbb{Q}
\usepackage{amssymb}
\usepackage{amsthm}
\usepackage{graphicx}
\usepackage{listings}
\usepackage{nameref}
\usepackage{systeme}
\usepackage{enumitem}
\newtheorem{lemma}{Lemma}
\begin{document}
\maketitle
\lstset{basicstyle=\ttfamily}

\section*{Ex 5}
Note: \textbf{I use $e$ as if it's polymorphic. It should be clear which set it
belongs to based on context}.
\begin{itemize}
\item [20.]
  % Do as a diagram project
\item [22.]
  $\{
  \begin{bmatrix}
    0 & -1 \\
    -1 & 0
  \end{bmatrix},
  \begin{bmatrix}
    1 & 0 \\
    0 & 1
  \end{bmatrix}
\}$
\item [41.]
  The proof heavily relies on $Theorem 5.14$.
  \begin{description}
  \item [Closed]
    Let $a,b \in \Phi[H]$. Then we have $a = \Phi(h)$, $b=\Phi(h')$
    for some $h,h' \in H$. $ab = \Phi(h)\Phi(h')=\Phi(hh') \in
    H$. $hh' \in H$ is true, by Theorem 5.14.
  \item [Identity]
        This can be done by showing that $\Phi(e)$is the identity
        element.

        $e$ of $G$ is in $H$ by Theorem 5.14.
        Let $a \in \Phi[H]$, then $a = \Phi(h)$ for some $h \in H$. $a
        \Phi(e) = \Phi(he) = \Phi(h) = a$.
  \item [Inverse]
    Let $a \in \Phi[H]$. $a=\Phi(h)$ for some $h\in H$. $H$ is a
    subgroup, so we have $h^{-1}$ by Theorem 5.14. Let $b=\Phi(h^{-1})$, then we get $ab
    = \Phi(h)\Phi(h^{-1}) = e$, $ba=\Phi(h^{-1})\Phi(h)=\Phi(e)=e$
  \end{description}
  By Theorem 5.14, since the above three properties hold, $\Phi(H)$ is
  a subgroup of $G'$.
\end{itemize}

\section*{Ex 6}
\begin{itemize}
\item [17.] $\{0,5,10,15,20,25\}$
\item [20.] Let $a = \frac{1+i}{\sqrt{2}}$.
  $a^2=i$, $a^3=\frac{i-1}{\sqrt{2}}$, $a^4=-1$,
  $a^5=\frac{-1-i}{\sqrt{2}}$, $a^5=\frac{-1-i}{\sqrt{2}}$, $a^6=-i$,
  $a^7=\frac{-i+1}{\sqrt{2}}$, $a^8=1$. The number of elements is $8$.
\item [56.]
  \begin{enumerate}[label=\alph*.]
  \item
    \begin{proof}
      Let $a$ and $b$ each be the generator of $H$ and $K$.  We claim
      that $S=<ab>$ is a cyclic subgroup of order $rs$.

    \begin{lemma}
      $c= a^m = b^n \rightarrow c = e$
    \end{lemma}
    \begin{proof}
      By Theorem 6.14,
      $ord(c)$ is a divisor of $r$ and $s$. Therefore $ord(c)$ has to
      be $1$ because otherwise $r$ and $s$ will have a common divisor
      greater than $1$, which contradicts the premiss that $r$ and $s$
      are relatively prime, and hence $c=e$.
    \end{proof}
    Suppose $(ab)^m = e$. Since the group is abelian, we can rewrite
    it as $a^m = b^{-m}$. By Lemma 1, $a^m=b^{-m}=e$. Since $ord(a)=r$
    and $ord(b)=s$, $r$ and $s$ both divides $m$. $gcd(r,s)=1$,
    $rs$ divides $m$, and thus $ord(ab)=rs$
  \end{proof}
  \end{enumerate}
\end{itemize}

\end{document}
