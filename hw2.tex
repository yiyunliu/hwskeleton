\documentclass{article}
\title{HW 3}
\date{2018/09/13}
\author{Yiyun Liu}
\usepackage{proof}
\usepackage{hyperref}
\usepackage{mathtools}
\usepackage{amsmath}
\usepackage{booktabs}
\usepackage{diagbox}
\usepackage[a4paper,total={6in,8in}]{geometry}
% For \mathbb{R} or \mathbb{Q}
\usepackage{amssymb}
\usepackage{amsthm}
\usepackage{graphicx}
\usepackage{listings}
\usepackage{nameref}
\usepackage{systeme}
\usepackage{enumitem}
\begin{document}
\maketitle
\lstset{basicstyle=\ttfamily}

\section*{Exercises 6}
\begin{itemize}
\item [32.]
  \begin{itemize}
  \item [e.] true.
    \begin{proof}
      Suppose $a$ and $b$ are elements of some cyclic group $G$. We have $a=g^m$ and $b=g^n$ for some element $g$. $ab = g^{m+n} = g^{n+m}=ba$.
    \end{proof}
  \item [f.] false\\
    \begin{tabular}{c | c c c c }
      $\cdot$&e&a&b&c\\
      \hline
      e&e&a&b&c\\
      a&a&e&c&b\\
      b&b&c&e&a\\
      c&c&b&a&e
    \end{tabular}
    The above group has order $4$ but not cyclic. 
  \item [g.] false.
    $1$ is a generator of $\mathbb{Z}_{20}$ but it's not a prime number.
  \item [h.] false. The set $\{0\}$ and $\mathbb{Z}^*$ are both groups under $*$, but their intersection is not even a non-empty set.
  \item [i.] true.
    \begin{proof}
      $e$ is clearly in $H \cap K$, since both $H$ and $K$ have neutral element inherited from $G$.

      Suppose $a \in H \cap K$ and $b \in H \cap K$. $a,b\in H$ and $a,b \in K$. Since the operation is closed under both $H$ and $K$, $ab \in H$ and $ab \in K$, and hence $ab \in H \cap K$.

      Suppose $a \in H \cap K$. $a \in H$ and $a \in K$, and thus $a$ has an inverse of $a$ in $H$ and an inverse of $a$ in $K$. $H$ and $K$ are subgroups of $G$, so the two inverses are inverses of $a$ in $G$. Because we know inverse of $a$ is unique in $G$, the two inverses are equal. Let's call this element $b$. $b \in H \cap K$ and $b$ satisfies the properties for inverse of $a$. $a$ has an inverse in $H \cap K$.
    \end{proof}
  \item [j.] true.
    \begin{proof}
      Let $g$ be a generator for the group, where $|\langle g \rangle|=k$. Because $k>2$, $k-1$ and $1$ are both relatively prime to $k$. By theorem 6.14, $g^{k-1}$ and $g$ are two generators of this group. They are clearly distinct, as proved in textbook.
    \end{proof}
  \end{itemize}
\end{itemize}

\section*{Exercises 8}
\begin{itemize}
\item [4.]
  \[\begin{pmatrix}
    1 & 2 & 3 & 4 & 5 & 6\\
    5 & 1 & 6 & 2 & 4 & 3
  \end{pmatrix}\]
\item [16.]
  $3P3 = 3! = 6$. One element is fixed, so we count only the permutation of the rest of three elements on three remaining slots.
\item [18.]
  \begin{enumerate}[label=\alph*.]
  \item
    \[\langle \rho_1 \rangle = \{\rho_0, \rho_1, \rho_2\}\]
    \[\langle \rho_2 \rangle = \{\rho_0, \rho_1, \rho_2\}\]
    \[\langle \mu_1 \rangle = \{\rho_0, \mu_1\}\]
  \item 
  \end{enumerate}
\end{itemize}


\end{document}
