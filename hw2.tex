\documentclass{article}
\title{HW 2}
\date{2018/09/13}
\author{Yiyun Liu}
\usepackage{proof}
\usepackage{hyperref}
\usepackage{mathtools}
\usepackage{amsmath}
\usepackage{booktabs}
\usepackage{diagbox}
\usepackage[a4paper,total={6in,8in}]{geometry}
% For \mathbb{R} or \mathbb{Q}
\usepackage{amssymb}
\usepackage{color}
\usepackage{amsthm}
\usepackage{graphicx}
\usepackage{listings}
\usepackage{nameref}
\usepackage{systeme}
\usepackage{enumitem}

\definecolor{mygreen}{rgb}{0,0.6,0}
\definecolor{mygray}{rgb}{0.5,0.5,0.5}
\definecolor{mymauve}{rgb}{0.58,0,0.82}



\lstset{ 
  backgroundcolor=\color{white},   % choose the background color; you must add \usepackage{color} or \usepackage{xcolor}; should come as last argument
  basicstyle=\footnotesize,        % the size of the fonts that are used for the code
  breakatwhitespace=false,         % sets if automatic breaks should only happen at whitespace
  breaklines=true,                 % sets automatic line breaking
  captionpos=b,                    % sets the caption-position to bottom
  commentstyle=\color{mygreen},    % comment style
  deletekeywords={...},            % if you want to delete keywords from the given language
  escapeinside={\%*}{*)},          % if you want to add LaTeX within your code
  extendedchars=true,              % lets you use non-ASCII characters; for 8-bits encodings only, does not work with UTF-8
  frame=single,	                   % adds a frame around the code
  keepspaces=true,                 % keeps spaces in text, useful for keeping indentation of code (possibly needs columns=flexible)
  keywordstyle=\color{blue},       % keyword style
  language=Octave,                 % the language of the code
  morekeywords={*,...},            % if you want to add more keywords to the set
  % numbers=left,                    % where to put the line-numbers; possible values are (none, left, right)
  numbersep=5pt,                   % how far the line-numbers are from the code
  numberstyle=\tiny\color{mygray}, % the style that is used for the line-numbers
  rulecolor=\color{black},         % if not set, the frame-color may be changed on line-breaks within not-black text (e.g. comments (green here))
  showspaces=false,                % show spaces everywhere adding particular underscores; it overrides 'showstringspaces'
  showstringspaces=false,          % underline spaces within strings only
  showtabs=false,                  % show tabs within strings adding particular underscores
  stepnumber=2,                    % the step between two line-numbers. If it's 1, each line will be numbered
  stringstyle=\color{mymauve},     % string literal style
  tabsize=2,	                   % sets default tabsize to 2 spaces
  title=\lstname                   % show the filename of files included with \lstinputlisting; also try caption instead of title
}
\begin{document}
\maketitle
\lstset{basicstyle=\ttfamily}
For many of the exercises, the computations were done with GNU
Octave. I'll put the code here as proof of work.
\section*{Exercise 2.2}
\begin{lstlisting}
>> M = [0.3157 0.947; 0.6314 0.1263]
>> [v,lambda] = eig(M)
v =

   0.81052  -0.73496
   0.58571   0.67811

lambda =

Diagonal Matrix

   1.00004         0
         0  -0.55804
>> v(:,1)
ans =
   0.81052
   0.58571
\end{lstlisting}

We are looking for the eigenvectors corresponding to the eigenvalue
$1$. Those vectors are solutions of the equation $Mp=p$ since $d=0$.

The octave code shows that $
\begin{pmatrix}
  0.81052\\
  0.58571
\end{pmatrix}
$ is one of such vector. In fact, multiply this vector with any
non-zero scalar gives us another vector which satisfies the same
equation. For example, the vector: $
\begin{pmatrix}
  1.6210\\
  1.1714
\end{pmatrix}
$

\section*{Exercise 2.5}
First, we set up the consumption matrix and the external demand vector.
\begin{lstlisting}
>> M = [0.22 0.16; 0.15 0.26]
M =

   0.22000   0.16000
   0.15000   0.26000

>> d = [120 150]'
d =

   120
   150
\end{lstlisting}
\begin{enumerate}[label=(\alph*)]
\item
\begin{lstlisting}
>> p = (eye(2)-M)^(-1)*d
p =

   203.90
   244.03

>> ceil(p)
ans =

   204
   245
\end{lstlisting}
  $ceil(p)$ is for the case where products are atomic and cannot be
  divided below $1$.
\item
\begin{lstlisting}
>> p - d
ans =

   83.905
   94.035

>> ceil(p-d)
ans =

   84
   95
\end{lstlisting}
  Internal demand is lower than external demand, which means it is
  relatively efficient (at least more efficient than the example given
  in textbook). 
\item
\begin{lstlisting}
M = 0.1 * M
M =

   0.022000   0.016000
   0.015000   0.026000

>> p = (eye(2)-M)^(-1)*d
p =

   125.25
   155.93

>> ceil(p)
ans =

   126
   156

>> p - d
ans =

   5.2504
   5.9330

>> ceil(p - d)
ans =

   6
   6
\end{lstlisting}
Much more efficient than the previous case. The internal demand is way
lower than external demand.
\item
\begin{lstlisting}
>> (eye(2)-M)^(-1)
ans =

   1.33767   0.28923
   0.27115   1.40998
\end{lstlisting}
  If external demand of product $1$ increase by $1$, the
  production of product $1$ increase by $1.33767$; if external demand
  of product $2$ increase by $1$, the production of product $1$
  increase by $0.28923$.
\end{enumerate}

\section*{Exercise 2.9}
\begin{lstlisting}
>> M = [0.1 0.06; 0.05 0.12]
M =

   0.100000   0.060000
   0.050000   0.120000

>> eye(2) + M
ans =

   1.100000   0.060000
   0.050000   1.120000

>> eye(2) + M + M^2
ans =

   1.113000   0.073200
   0.061000   1.137400

>> eye(2) + M + M^2 + M^3
ans =

   1.114960   0.075564
   0.062970   1.140148

>> eye(2) + M + M^2 + M^3 + M^4
ans =

   1.115274   0.075965
   0.063304   1.140596

>> eye(2) + M + M^2 + M^3 + M^4 + M^5
ans =

   1.115326   0.076032
   0.063360   1.140670

>> eye(2) + M + M^2 + M^3 + M^4 + M^5 + M^6
ans =

   1.115334   0.076043
   0.063370   1.140682
\end{lstlisting}
\begin{enumerate}[label=(\alph*)]
\item $i=6$. 
\item \(
    \begin{pmatrix}
   1.115334 &   0.076043 \\
   0.063370  & 1.140682
    \end{pmatrix}
\)
\item $(I-M)^{-1} \approx     \begin{pmatrix}
   1.115334 &   0.076043 \\
   0.063370  & 1.140682
    \end{pmatrix}$
\end{enumerate}

\section*{Exercise 2.17}
\begin{enumerate}[label=(\alph*)]
\item Use the equality $M = - (((I-M)^{-1})^{-1}-I)$.
\item
\begin{lstlisting}
>> IM = [1.06 0.02 0.04; 0.02 1.06 0.09; 0.11 0.02 1.03]
IM =

   1.060000   0.020000   0.040000
   0.020000   1.060000   0.090000
   0.110000   0.020000   1.030000

>> M = -((IM^(-1))-eye(3))M =

   0.0526171   0.0172093   0.0352878
   0.0093000   0.0548769   0.0822224
   0.1009962   0.0165140   0.0237611
\end{lstlisting}
\end{enumerate}

\section*{Exercise 2.19}

\begin{enumerate}[label=(\alph*)]
\item $M =
\begin{pmatrix}
  0.1 & x \\
  0.2 & 0.05
\end{pmatrix}
$\\
$I-M =
\begin{pmatrix}
  0.9 & 1-x \\
  0.8 & 0.95
\end{pmatrix}
$

In order for the equation $(I-M)p = d$ to have solution for arbitrary
$d$, we need $I-M$ to be invertible. Therefore, we need $det(I-M) \neq
0$.
\begin{equation}
  \begin{split}
    det(I-M)
    &= 0.9 \cdot 0.95 - (1-x)\cdot 0.8 \\
    &= 0.055 + 0.8x
  \end{split}
\end{equation}
Since $x$ is always non-negative, $det(I-M)$ is never $0$, and hence
any non-negative $x$ would give solution to arbitrary $d$.
\item $M =
\begin{pmatrix}
  0.1 & x \\
  y & 0.05
\end{pmatrix}
$\\
$I-M =
\begin{pmatrix}
  0.9 & 1-x \\
  1-y & 0.95
\end{pmatrix}
$

Simplify the inequality, we get:
$det(I-M)= -x y+y+x-0.145 \neq 0$.
Of course, $x$ and $y$ also need to satisfy the implicit constraint
that both are non-negative.
\end{enumerate}



\end{document}


